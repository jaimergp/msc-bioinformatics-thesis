\chapter{Conclusions}



%___________________________________________________________________________

\section*{\phantomsection%
  Conclusions%
  \addcontentsline{toc}{section}{Conclusions}%
  \label{conclusions}%
}

GAUDI has proved to be a valid proof of concept. Simplistic energy calculation terms do generate good results if combined with the appropriate restraints, which is one of the most powerful strengths of this software. It can help molecular designers with a rapid sketching tool that will produce feasible in results in a reasonable amount of time.


%___________________________________________________________________________

\section*{\phantomsection%
  Further work%
  \addcontentsline{toc}{section}{Further work}%
  \label{further-work}%
}
%
\begin{itemize}

\item Further development of the spatial exploration engine: improve free docking results, global protein flexibility

\item More objectives: Metal coordination geometries prediction (not just visual aid)

\item QM/MM minimization interface from GaudiView: direct refinement from the GUI

\item Performance improvements: parallelization, code polish

\item Custom %
\raisebox{1em}{\hypertarget{id2}{}}\hyperlink{id1}{\textbf{\color{red}*}}.frcmod inputs as energy terms

\end{itemize}


