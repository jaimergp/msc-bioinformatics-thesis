\chapter{Conclusions}%

The results of my Masters in Science has been named after the famous Catalonian architect, Antoni Gaudí. GAUDI\textsubscript{ASM} stands for \textbf{G}enetic \textbf{A}lgorithms for \textbf{U}niversal \textbf{D}esign \textbf{I}nference and \textbf{A}tomic \textbf{S}cale \textbf{M}odelling. As a result, I can proudly present a novel platform that aims to satisfy increasingly demanding areas of molecular design. It will do so by providing the researchers with a powerful multi-objective optimization engine to explore the huge search space that chemobiological design problems usually present.

Despite being at its infancy, GAUDI\textsubscript{ASM} has proved to be a valid proof of concept. It will help both experimentalists and theoreticians with a rapid sketching tool that will produce feasible in results in a reasonable amount of time. After all, it has already shed light on two experiments where no existent tools could hardly help with.


%___________________________________________________________________________

\section*{\phantomsection%
  Further work%
  \addcontentsline{toc}{section}{Further work}%
  \label{further-work}%
}

After only six months of work, GAUDI\textsubscript{ASM} cannot be considered a finished product. We are already making big steps towards a stable 1.0 version, which will include some large improvements. This is a non-exhaustive list of such milestones.
%
\begin{itemize}

\item Further development of the spatial exploration engine: improve free docking results and benchmarking, global protein flexibility

\item More objectives, like a transition metal coordination geometry predictive engine

\item An integrative processor that could deal with MM, QM/MM to QM in a pipeline manner, implementing easy refinement tools directly into GaudiView as the starting point of a vertical platform

\item Custom *.frcmod inputs as energy terms

\item Performance improvements: thread parallelization, code polish

\end{itemize}